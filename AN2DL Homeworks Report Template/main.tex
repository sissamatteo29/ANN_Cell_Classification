\documentclass[11pt]{article}
\usepackage[utf8]{inputenc}
\usepackage[english]{babel}
\usepackage{amsmath}
\usepackage{graphicx}
\usepackage{float}
\usepackage{lipsum}
\usepackage{multicol}
\usepackage{xcolor}
\usepackage{tabularx}
\usepackage{booktabs}
\usepackage{hyperref}
\newcolumntype{Y}{>{\centering\arraybackslash}X}
\usepackage[left=2.00cm, right=2.00cm, top=2.00cm, bottom=2.00cm]{geometry}

\title{AN2DL First Homework Report}

\begin{document}
    
    \begin{figure}[H]
        \raggedright
        \includegraphics[scale=0.4]{polimi.png} \hfill \includegraphics[scale=0.3]{airlab.jpeg}
    \end{figure}
    
    \vspace{5mm}
    
    \begin{center}
        % Select between First and Second
        {\Large \textbf{AN2DL - First Homework Report}}\\
        \vspace{2mm}
        % Change with your Team Name
        {\Large \textbf{NeuralDropouts}}\\
        \vspace{2mm}
        % Team Members Information
        {\large Pinar Erbil,}
        {\large Sergio Pardo,}
        {\large Angela Remolina,}
        {\large Matteo Sissa}\\
        \vspace{2mm}
        % Codabench Nicknames
        {perbil,}
        {sergiopardo,}
        {angelaremolina,}
        {Codabench Nickname4}\\
        \vspace{2mm}
        % Matriculation Numbers
        {Matricola1,}
        {10985243,}
        {Matricola3,}
        {Matricola4}\\
        \vspace{5mm}
        \today
    \end{center}    
    \vspace{5mm}
    
    \begin{multicols}{2}
        \noindent \textit{Note: The following sections  represent a suggested structure. Feel free to adapt them to better suit your specific project needs.}
        
        \section{Introduction}
        Artificial intelligence has revolutionized the world we live in, the way we look at things, and the problems we aim to solve. From healthcare to transportation, AI-powered systems are enhancing efficiency, accuracy, and accessibility in ways previously unimaginable. It enables doctors to diagnose diseases faster, automates mundane tasks to free up human creativity, and powers technologies like self-driving cars and smart assistants among other advancements. 
        \newline

        Additionally, technologies like deep learning and deep neural networks have allowed huge improvements in the computer vision field. Now, systems with super human performance have taken over and are now the industry standard.
        \newline
        
        Medical Image Analysis falls in-between these two areas, it implements cutting edge deep learning models to boost the capability humanity has of studying our body and its composition. This is the context of the present project. Its aim is to \textbf{classify 96x96 RGB images of blood cells among 8 possible classes}. Therefore, the goals our team set were; 
        \begin{enumerate}
            \item Understand the given dataset.
            \item Build a deep learning model that is able to make predictions on the dataset. 
            \item Achieve a prediction accuracy of at least 80\% .
        \end{enumerate}

        In order to achieve this, we devised the following approach:
        \begin{itemize}
            \item Explore the data with statistics and some visualization techniques that will allow the detection of outliers and understand the structure of the data.
            \item Given the size of the dataset, perform data augmentation to have a wider dataset that will allow the model to be more robust when doing predictions.
            \item Build a model from scratch or by using \textbf{transfer learning} techniques with pre-trained models to train it on the training set.
            \item Test all these models in the CodaBench platform to evaluate their performance and assess possible changes.
        \end{itemize}
        .

        
        \section{Problem Analysis}
        Here you can discuss your initial analysis of the problem. Consider including:
        \begin{enumerate}
            \item Dataset characteristics
            \item Main challenges
            \item Initial assumptions
        \end{enumerate}

        \noindent If you need to reference papers, use the citation command: Recent work~\cite{lecun2015deep} suggests..."

        \section{Method}
        This section should detail your approach. You can use equations to explain your methodology. For example, a simple model representation:
        \begin{equation}
            \label{eq:model}
            f(x) = \text{softmax}(Wx + b)
        \end{equation}

        \noindent Or a more complex loss function:
        \begin{equation}
            \label{eq:loss}
            \mathcal{L} = -\frac{1}{N}\sum_{i=1}^{N} y_i\log(\hat{y}_i)
        \end{equation}

        \noindent Reference these equations in your text, like:``As shown in equation~\ref{eq:model}..."

        \section{Experiments}
        For your experiments, you might want to present your results in tables. Here's an example of a wide table comparing different models:

        \begin{table*}[t]
            \centering
            \setlength{\tabcolsep}{3pt}
            \caption{An example of wide table. Best results are highlighted in \textbf{bold}.}
            \begin{tabularx}{\textwidth}{lYYYc}
                \toprule
                Model & Accuracy & Precision & Recall & ROC AUC\\
                \midrule
                VGG18         &  72.20 $\pm$ 3.06    &   94.95 $\pm$ 0.52     &   86.95 $\pm$ 0.55    &   80.16 $\pm$ 0.81\\
                Custom Model        &  27.71 $\pm$ 3.19    &   75.70 $\pm$ 1.07     &   55.75 $\pm$ 2.16    &   36.60 $\pm$ 1.26\\
                ResNet18    &  \textbf{89.24 $\pm$ 2.38}    &   \textbf{95.54 $\pm$ 0.49}     &   \textbf{93.43 $\pm$ 1.30}    &   \textbf{91.68 $\pm$ 0.71}\\
                \bottomrule
            \end{tabularx}
            \label{tab:Performance}
        \end{table*}

        \noindent For more specific measurements, you might use a narrower table:
    
        \begin{table}[H]
            \centering
            \setlength{\tabcolsep}{3pt}
            \caption{An example of table. Best results may be highlighted in \textbf{bold}.}
            \begin{tabularx}{\linewidth}{lY}
                \toprule
                Time [$\mu$s] & Distance [mm]\\
                \midrule
                22$\pm$4 & 8$\pm$1\\
                17$\pm$3 & 7$\pm$1\\
                15$\pm$3 & 6$\pm$1\\
                13$\pm$2 & 5$\pm$1\\
                10$\pm$2 & 4$\pm$1\\
                8$\pm$2 & 3$\pm$1\\
                5$\pm$1 & 2$\pm$1\\
                37$\pm$1 & 1$\pm$1\\
                \bottomrule
            \end{tabularx}
            \label{tb:Measurements}
        \end{table}

        \noindent You can also include figures to visualise your results:
        \begin{figure}[H]
            \centering
            \includegraphics[width=0.75\linewidth]{random.jpeg}
            \caption{Example figure showing [describe what the figure shows]}
            \label{fig:results}
        \end{figure}

        \noindent Reference figures using like:``As shown in Figure~\ref{fig:results}..."

        \section{Results}
        Present your main findings here. You might want to:
        \begin{itemize}
            \item Compare your results with baselines
            \item Highlight key achievements using \textbf{bold text}
            \item Explain any unexpected outcomes
        \end{itemize}

        \section{Discussion}
        In this section, analyse your results critically. Consider:
        \begin{itemize}
            \item Strengths and weaknesses
            \item Limitations and assumptions
        \end{itemize}

        \section{Conclusions}
        Summarise your work and discuss potential future directions. This is where you can:
        \begin{itemize}
            \item Restate main contributions
            \item Suggest improvements
            \item Propose future work
        \end{itemize}

        \bibliography{references}
        \bibliographystyle{abbrv}
    
    \end{multicols}
\end{document}