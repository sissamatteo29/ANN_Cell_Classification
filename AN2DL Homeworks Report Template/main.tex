\documentclass[11pt]{article}
\usepackage[utf8]{inputenc}
\usepackage[english]{babel}
\usepackage{amsmath}
\usepackage{graphicx}
\usepackage{float}
\usepackage{lipsum}
\usepackage{multicol}
\usepackage{xcolor}
\usepackage{tabularx}
\usepackage{booktabs}
\usepackage{hyperref}
\newcolumntype{Y}{>{\centering\arraybackslash}X}
\usepackage[left=2.00cm, right=2.00cm, top=2.00cm, bottom=2.00cm]{geometry}

\title{AN2DL Reports Template}

\begin{document}
    
    \begin{figure}[H]
        \raggedright
        \includegraphics[scale=0.4]{polimi.png} \hfill \includegraphics[scale=0.3]{airlab.jpeg}
    \end{figure}
    
    \vspace{5mm}
    
    \begin{center}
        % Select between First and Second
        {\Large \textbf{AN2DL - First/Second Homework Report}}\\
        \vspace{2mm}
        % Change with your Team Name
        {\Large \textbf{Team Name}}\\
        \vspace{2mm}
        % Team Members Information
        {\large Name1 Surname1,}
        {\large Name2 Surname2,}
        {\large Name3 Surname3,}
        {\large Name4 Surname4}\\
        \vspace{2mm}
        % Codabench Nicknames
        {Codabench Nickname1,}
        {Codabench Nickname2,}
        {Codabench Nickname3,}
        {Codabench Nickname4}\\
        \vspace{2mm}
        % Matriculation Numbers
        {Matricola1,}
        {Matricola2,}
        {Matricola3,}
        {Matricola4}\\
        \vspace{5mm}
        \today
    \end{center}    
    \vspace{5mm}
    
    \begin{multicols}{2}
        \noindent \textit{Note: The following sections represent a suggested structure. Feel free to adapt them to better suit your specific project needs.}
        
        \section{Introduction}
        In this section, you should present your project's context and objectives. You might want to:
        \begin{itemize}
            \item Define the problem (\textit{you may use italics to highlight definitions})
            \item State your goals (\textbf{emphasise key points with bold})
            \item Outline your approach
        \end{itemize}

        \noindent For instance, you might write: ``This project focuses on \textit{image classification} using \textbf{deep learning} techniques."
        
        \section{Problem Analysis}
        Here you can discuss your initial analysis of the problem. Consider including:
        \begin{enumerate}
            \item Dataset characteristics
            \item Main challenges
            \item Initial assumptions
        \end{enumerate}

        \noindent If you need to reference papers, use the citation command: Recent work~\cite{lecun2015deep} suggests..."

        \section{Method}
        This section should detail your approach. You can use equations to explain your methodology. For example, a simple model representation:
        \begin{equation}
            \label{eq:model}
            f(x) = \text{softmax}(Wx + b)
        \end{equation}

        \noindent Or a more complex loss function:
        \begin{equation}
            \label{eq:loss}
            \mathcal{L} = -\frac{1}{N}\sum_{i=1}^{N} y_i\log(\hat{y}_i)
        \end{equation}

        \noindent Reference these equations in your text, like:``As shown in equation~\ref{eq:model}..."

        \section{Experiments}
        For your experiments, you might want to present your results in tables. Here's an example of a wide table comparing different models:

        \begin{table*}[t]
            \centering
            \setlength{\tabcolsep}{3pt}
            \caption{An example of wide table. Best results are highlighted in \textbf{bold}.}
            \begin{tabularx}{\textwidth}{lYYYc}
                \toprule
                Model & Accuracy & Precision & Recall & ROC AUC\\
                \midrule
                VGG18         &  72.20 $\pm$ 3.06    &   94.95 $\pm$ 0.52     &   86.95 $\pm$ 0.55    &   80.16 $\pm$ 0.81\\
                Custom Model        &  27.71 $\pm$ 3.19    &   75.70 $\pm$ 1.07     &   55.75 $\pm$ 2.16    &   36.60 $\pm$ 1.26\\
                ResNet18    &  \textbf{89.24 $\pm$ 2.38}    &   \textbf{95.54 $\pm$ 0.49}     &   \textbf{93.43 $\pm$ 1.30}    &   \textbf{91.68 $\pm$ 0.71}\\
                \bottomrule
            \end{tabularx}
            \label{tab:Performance}
        \end{table*}

        \noindent For more specific measurements, you might use a narrower table:
    
        \begin{table}[H]
            \centering
            \setlength{\tabcolsep}{3pt}
            \caption{An example of table. Best results may be highlighted in \textbf{bold}.}
            \begin{tabularx}{\linewidth}{lY}
                \toprule
                Time [$\mu$s] & Distance [mm]\\
                \midrule
                22$\pm$4 & 8$\pm$1\\
                17$\pm$3 & 7$\pm$1\\
                15$\pm$3 & 6$\pm$1\\
                13$\pm$2 & 5$\pm$1\\
                10$\pm$2 & 4$\pm$1\\
                8$\pm$2 & 3$\pm$1\\
                5$\pm$1 & 2$\pm$1\\
                37$\pm$1 & 1$\pm$1\\
                \bottomrule
            \end{tabularx}
            \label{tb:Measurements}
        \end{table}

        \noindent You can also include figures to visualise your results:
        \begin{figure}[H]
            \centering
            \includegraphics[width=0.75\linewidth]{random.jpeg}
            \caption{Example figure showing [describe what the figure shows]}
            \label{fig:results}
        \end{figure}

        \noindent Reference figures using like:``As shown in Figure~\ref{fig:results}..."

        \section{Results}
        Present your main findings here. You might want to:
        \begin{itemize}
            \item Compare your results with baselines
            \item Highlight key achievements using \textbf{bold text}
            \item Explain any unexpected outcomes
        \end{itemize}

        \section{Discussion}
        In this section, analyse your results critically. Consider:
        \begin{itemize}
            \item Strengths and weaknesses
            \item Limitations and assumptions
        \end{itemize}

        \section{Conclusions}
        Summarise your work and discuss potential future directions. This is where you can:
        \begin{itemize}
            \item Restate main contributions
            \item Suggest improvements
            \item Propose future work
        \end{itemize}

        \bibliography{references}
        \bibliographystyle{abbrv}
    
    \end{multicols}
\end{document}